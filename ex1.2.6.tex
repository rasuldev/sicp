\documentclass[a4paper,10pt]{article}
\usepackage[T2A]{fontenc}
\usepackage[utf8]{inputenc}
\usepackage{amsmath, amsfonts, amssymb, amsthm, dsfont}
\usepackage{amscd}
\usepackage[russian]{babel}
%\usepackage{pscyr}
\usepackage{cmap}

\textwidth=16.5cm
\hoffset=-2.5cm
\textheight=25.5cm
\voffset=-2.5cm


\righthyphenmin=2
\parindent=0mm
\parskip=3mm

\newcommand{\suml}{\sum\limits}
\newcommand{\intl}{\int\limits}


\DeclareMathOperator*{\esssup}{ess\,sup}
\DeclareMathOperator*{\essinf}{ess\,inf}

%========Библиография=========
\makeatletter
\renewcommand{\@biblabel}[1]{#1.}
\makeatother

\bibliographystyle{utf8gost705u}
%================


\newtheorem{lemma}{Лемма}
\newtheorem{theorem}{Теорема}
\newtheorem*{remark}{Замечание}
%\renewcommand{\theremark}{}
\newtheorem{theoremrus}{Теорема}
\newtheorem{lemmarus}{Лемма}
\newtheorem{statementrus}{Утверждение}
\newtheorem{remarkrus}{Замечание}

\newcommand{\highphantom}{\vphantom{\dfrac{1}{1}}}
\renewcommand{\thesubsubsection}{\arabic{subsubsection}.}

%\renewcommand{\thelemma}{\arabic{lemma}}
%\renewcommand{\proofname}{Доказательство}
\renewcommand{\arraystretch}{2}

\begin{document}
(define (expmod base exp m)\ldots:

3rd condition. We can calc $b^{2n} \bmod m$ using $b^n \bmod m$. Indeed, let
\begin{equation}\label{bn}
b^n = mq + r.
\end{equation}
Note, that $r = b^n \bmod m$. If we square both parts of \eqref{bn}, we'll get:
\begin{equation*}
  b^{2n} = m^2q^2+2mqr + r^2.
\end{equation*}
It's clear that first two summands are divided by m, so remainder will depend only on $r^2$:
\begin{equation*}
    b^{2n} \bmod m = r^2 \bmod m = (b^n \bmod m)^2 \bmod m.
\end{equation*}

4th condition. Let $b^{n-1}=mq+r$. Then $b^n=bmq+rb$, and we have
\begin{equation*}
  b^n \bmod m = rb \bmod m = ((b^{n-1} \bmod m)b) \bmod m.
\end{equation*}

\end{document}
